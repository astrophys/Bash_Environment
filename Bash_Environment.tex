% Compile by directly running 
%   pdflatex Slurm_Intro.tex 
%
%
%
%
%




\documentclass{beamer}
\usetheme{Copenhagen}
\usepackage[utf8]{inputenc}


%\usepackage{graphicx}
%\usepackage{subfigure}
%\usepackage{multimedia}
\usepackage{times}  % fonts are up to you
\usepackage{graphics}
\usepackage{amsmath}
\usepackage{media9}
\usepackage{hyperref}
\usepackage{psfrag}
\usepackage{pdfpages}
\usepackage{listings}


\setbeamertemplate{bibliography item}[text]
\newcommand{\customcite}[1]{\citeauthor{#1}, \citeyear{#1}}
\newcommand\smallFont{\fontsize{8}{7.2}\selectfont}   %Change font size.
\newcommand\mCite[1]{[\cite{#1}, \citetitle{#1}]}  %Prints name and title
\newcommand\FrameText[1]{
\begin{textblock*}{\paperwidth}(0pt,\textheight)
	\vspace{1.0cm}
    \raggedleft \smallFont #1
\end{textblock*}}

%Get rid of ugly copenhagen default symbol for enumerate
\setbeamertemplate{enumerate items}[default]   


% Create code text
% https://tex.stackexchange.com/questions/65291/code-snippet-in-text
\definecolor{codegray}{gray}{0.9}
\newcommand{\code}[1]{\colorbox{codegray}{\texttt{#1}}}



%Information to be included in the title page:
\title{Bash Environment and Managing Software}
\author{Ali Snedden}
\institute{Nationwide Children's Hospital}
\date{May 3, 2019}
 
 
 
\begin{document}
 
\frame{\titlepage}




\begin{frame}
\frametitle{How to Connect}
Windows:
\begin{itemize}
    \item Open PuTTY
    \item Window Session $\Rightarrow$ Host Name field : username@10.70.250.101
    \item Click ``Open" to log in.
    \item Enter password
\end{itemize}

Mac:
\begin{itemize}
    \item Open Terminal (Finder $\Rightarrow$ Utilities $\Rightarrow$ Terminal)
    \item \code{ssh -X username@10.70.250.101}
\end{itemize}
\end{frame}



\begin{frame}
\frametitle{Bash}
\begin{itemize}
    \item Environmental variables control what commands/libraries are available
    \pause
    \item E.g.
    \pause
    \begin{enumerate}
        \item \code{PATH} : Location shell looks for commands / executables
            \begin{itemize}
                \item Try : \code{echo \$PATH}
            \end{itemize}
        \pause
        \bigskip
        \item \code{PYTHONPATH} : Location python looks for modules (libraries).
            \begin{itemize}
                \item Try : \code{echo \$PYTHONPATH}
            \end{itemize}
        \bigskip
        \pause
        \item \code{LD\_LIBRARY\_PATH} : Where shell looks for shared libraries.
            \begin{itemize}
                \item Try : \code{echo \$LD\_LIBRARY\_PATH}
            \end{itemize}
        \pause
    \bigskip
    \item See all environmental variables try : \code{env | less}
    \end{enumerate}
\end{itemize}
\end{frame}


\begin{frame}
\frametitle{Modules}
\begin{itemize}
    \item We use modules to manage software on Baker.
    \pause
    \bigskip
    \item Try:
    \pause
    \begin{itemize}
        \item \code{module avail}
        \bigskip
        \pause
        \item \code{module load python/3.6.5}
        \bigskip
        \pause
        \item \code{module list}
        \bigskip
        \pause
        \item \code{module purge}
    \end{itemize}
\end{itemize}
\end{frame}


\begin{frame}
\frametitle{Modules}
\begin{itemize}
    \item Modules installed here : \code{/opt/}
    \pause
    \bigskip
    \item Modules files here : \code{/act/modulefiles/}
\end{itemize}
\end{frame}



%%% Make this a section?



\begin{frame}
\frametitle{Python Example}
\begin{itemize}
    \item Look at \code{/act/modulefiles/python/3.6.5}
    \pause
    \item Notice how \code{PATH} and \code{PYTHONPATH} are pre-pended.
    \pause
    \item Try:
    \begin{itemize}
        \item \code{module purge}
        \pause
        \item Check \code{PATH}, \code{PYTHONPATH}.
        \pause
        \item \code{which python}
        \pause
        \item \code{module load python/3.6.5}
        \pause
        \item Check how \code{PATH}, \code{PYTHONPATH} change.
        \pause
        \item \code{which python}
    \end{itemize}
\end{itemize}
\end{frame}


\begin{frame}[fragile]
\frametitle{Python Example}
With the system python (\code{/usr/bin/python}):
\begingroup
\scriptsize
\begin{lstlisting}[backgroundcolor = \color{codegray}, language = Bash, showstringspaces=false]
$ module purge
$ python
Python 2.6.6 (r266:84292, Aug 18 2016, 15:13:37)
[GCC 4.4.7 20120313 (Red Hat 4.4.7-17)] on linux2
Type "help", "copyright", "credits" or "license" for more info
>>> import numpy
>>> numpy.__file__
'/usr/lib64/python2.6/site-packages/numpy/__init__.pyc'
>>> exit()
\end{lstlisting}
\endgroup
\end{frame}


\begin{frame}[fragile]
\frametitle{Python Example}
With the system python/3.6.5 module :
\begingroup
\scriptsize
\begin{lstlisting}[backgroundcolor = \color{codegray}, language = Bash, showstringspaces=false]
$ module load python/3.6.5
License File : see https://docs.anaconda.com/
$ python
Python 3.6.5 |Anaconda, Inc.| (default, Apr 29 2018, 16:14:56)
[GCC 7.2.0] on linux
Type "help", "copyright", "credits" or "license" for more info
>>> import numpy
>>> numpy.__file__
'/opt/python/3.6.5/lib/python3.6/site-packages/numpy/__init__.py'
>>> exit()
\end{lstlisting}
\endgroup
\end{frame}


\begin{frame}
\frametitle{Python Example}
\begin{itemize}
    \item Notice when using \code{python/3.6.5} module, numpy is located within the directory \code{/opt/python/3.6.5/lib/python3.6/site-packages} specified in \code{PYTHONPATH} 
    \bigskip
    \pause
    \item Many different software utilize environmental variables in a fashion like Python does.
\end{itemize}
\end{frame}

%%% Make this a section?

\begin{frame}
\frametitle{Libraries}
\begin{itemize}
    \item Libraries save time by providing mechanism to use another person's code.
    \bigskip
    \pause
    \item There are people who've already invented the wheel, so use it.
    \bigskip
    \pause
    \item Powerful, optimized libraries include \code{glibc.so}, \code{lapack.so}, etc.
    \bigskip
    \pause
    \item All software use libraries. Important concept in managing your software.
\end{itemize}
\end{frame}


% see : https://stackoverflow.com/questions/12237282/whats-the-difference-between-so-la-and-a-library-files/12237595
\begin{frame}
\frametitle{Libraries}
\begin{itemize}
    \item Shared Libraries (ending with \code{.so})
    \pause
    \begin{itemize}
        \item Pros : 
        \begin{itemize}
            \item Memory footprint is smaller (multiple programs can use same lib simultaneously)
            \pause
            \item Libraries loaded at run time. Specified by \code{/etc/ld.so.conf.d/} and \code{LD\_LIBRARY\_PATH}
            \bigskip
            \pause
        \end{itemize}
        \item Cons : 
        \begin{itemize}
            \item Users must install and maintain dependant shared libraries (i.e. 'dependency hell')
            \pause
            \item Can get surprised by subtle version differences.
        \end{itemize}
        \bigskip
    \pause
    \end{itemize}
    \item Static Libraries (ending with \code{.a})
    \pause
    \begin{itemize}
        \item Pros : 
        \begin{itemize}
            \item Library is same version as software was built with
            \bigskip
            \pause
        \end{itemize}
        \item Cons : 
        \begin{itemize}
            \item Larger memory footprint.
        \end{itemize}
    \end{itemize}
\end{itemize}
\end{frame}


\begin{frame}[fragile]
\frametitle{Libraries Example}
%Trivial library example 
\begin{itemize}
    \item \code{headers.h} : 
    \begingroup
    \scriptsize
    \begin{lstlisting}[backgroundcolor = \color{codegray}, language = C, showstringspaces=false]
    #ifndef HEADERS
    #define HEADERS
    int add(int a, int b);
    int subtract(int a, int b);
    #endif
    \end{lstlisting}
    \endgroup
    
    \item \code{add.c} : 
    \begingroup
    \scriptsize
    \begin{lstlisting}[backgroundcolor = \color{codegray}, language = C, showstringspaces=false]
    #include <stdio.h>
    #include "headers.h"
    int add(int a, int b){
        print("Hello from add() in lib.1");
        return(a + b);
    }
    \end{lstlisting}
    \endgroup
    
    
    \item \code{subtract.c} : 
    \begingroup
    \scriptsize
    \begin{lstlisting}[backgroundcolor = \color{codegray}, language = C, showstringspaces=false]
    #include <stdio.h>
    #include "headers.h"
    int subtract(int a, int b){
        print("Hello from subtract() in lib.1");
        return(a - b);
    }
    \end{lstlisting}
    \endgroup
\end{itemize}
\end{frame}


\begin{frame}[fragile]
\frametitle{Libraries Example}
\begin{itemize}
    \item Copy 
    \begingroup
    \scriptsize
    \begin{lstlisting}[backgroundcolor = \color{codegray}, language = Bash, showstringspaces=false]
    $ cp -rp /gpfs0/scratch/Bash_Environment/codes/ ~/Scratch/
    \end{lstlisting}
    \endgroup
    \pause

    \item Navigate to \code{codes/} and compile with \code{./compile.sh} 
    \pause

    \item Generates exectuable \code{program} which uses \code{libmymath.so}
    \pause

    \item Library not found!
    \begingroup
    \scriptsize
    \begin{lstlisting}[backgroundcolor = \color{codegray}, language = Bash, showstringspaces=false]
    $ ldd program
	linux-vdso.so.1 =>  (0x00007ffc10f70000)
	libmymath.so => not found
	libc.so.6 => /lib64/libc.so.6 (0x00002b6c0f20d000)
	/lib64/ld-linux-x86-64.so.2 (0x000056347b79f000)
    \end{lstlisting}
    \endgroup
\end{itemize}
\end{frame}


\begin{frame}[fragile]
\frametitle{Libraries Example}
\begin{itemize}
    \item Try : 
    \begingroup
    \tiny
    \begin{lstlisting}[backgroundcolor = \color{codegray}, language = Bash, showstringspaces=false]
    $ export LD_LIBRARY_PATH=/some/path/codes/lib.1/:$LD_LIBRARY_PATH
    $ ldd program
	linux-vdso.so.1 =>  (0x00007ffc432f6000)
	libmymath.so => /some/path/codes/lib.1/libmymath.so (0x00002b64eba91000)
	libc.so.6 => /lib64/libc.so.6 (0x00002b64ebcb5000)
	/lib64/ld-linux-x86-64.so.2 (0x0000555da7d0b000)
    \end{lstlisting}
    \endgroup

    \pause
    \item Try : 
    \begingroup
    \tiny
    \begin{lstlisting}[backgroundcolor = \color{codegray}, language = Bash, showstringspaces=false]
    $ export LD_LIBRARY_PATH=/some/path/codes/lib.2/:$LD_LIBRARY_PATH
    $ ldd program
	linux-vdso.so.1 =>  (0x00007ffc432f6000)
	libmymath.so => /some/path/codes/lib.2/libmymath.so (0x00002af711fe1000)
	libc.so.6 => /lib64/libc.so.6 (0x00002b64ebcb5000)
	/lib64/ld-linux-x86-64.so.2 (0x0000555da7d0b000)
    \end{lstlisting}
    \endgroup

    \pause
    \item Now run : 
    \begingroup
    \tiny
    \begin{lstlisting}[backgroundcolor = \color{codegray}, language = Bash, showstringspaces=false]
    $ ./program
    9 + 2 = 11
    9 - 2 = 7
    \end{lstlisting}
    \endgroup
\end{itemize}
\end{frame}


\begin{frame}[fragile]
\frametitle{Libraries Example}
    \begingroup
    \scriptsize
    \begin{lstlisting}[backgroundcolor = \color{codegray}, language = Bash, showstringspaces=false]
    export LD_LIBRARY_PATH=/some/path/codes/lib.2/:$LD_LIBRARY_PATH
    \end{lstlisting}
    \endgroup

    \begin{itemize}
        \item \code{\scriptsize export} : makes variable propagate to daughter shell
        \pause
        \bigskip
        \item \code{\scriptsize \$LD\_LIBRARY\_PATH} : at end so we don't lose other libs we care about
        \pause
        \bigskip
        \item \code{\scriptsize /some/path/codes/lib.2/} : prepended because the first lib found is used
    \end{itemize}
\end{frame}



% Include slides on .bashrc / .profile
\begin{frame}
\frametitle{.bashrc and .profile}
\code{.bashrc} and \code{.profile}
\bigskip
\begin{itemize}
    \item One (or both) run at the beginning of shell session.
    \bigskip
    \pause
    \item Can put \code{echo "Hello from X"} at top to determine which runs when you log in.
\end{itemize}
\end{frame}


% Include slides on .bashrc / .profile
\begin{frame}
\frametitle{.bashrc and .profile}
Can put things in it like:
\bigskip
\begin{itemize}
    \item Modules, e.g. \code{module load python/3.5.6}
    \bigskip
    \pause
    \item Set environmental variables, e.g. \code{module load python/3.5.6}
    \pause
    \bigskip
    \item Aliases, e.g. \code{alias ll='ls -l'}
\end{itemize}
\end{frame}



\end{document}
